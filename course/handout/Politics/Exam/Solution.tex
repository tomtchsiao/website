\documentclass[14pt,a4paper,addpoints]{exam}
\usepackage{graphicx}
\usepackage{zhnumber}
\usepackage{ctex}
\usepackage{hyperref}
\usepackage{xcolor}
\usepackage{booktabs}
\usepackage{ifthen}
\usepackage{amsmath, amsthm, amssymb}
\pointname{ 分}
\pointformat{(\thepoints)}
\renewcommand{\thequestion}{\arabic{question}}
\renewcommand{\questionlabel}{\thequestion .}
\renewcommand{\thepartno}{\arabic{partno}}
\renewcommand{\partlabel}{ (\thepartno) }
\renewcommand{\thesubpart}{}
\renewcommand{\subpartlabel}{\thesubpart.}
\renewcommand{\thesubsubpart}{}
\renewcommand{\subsubpartlabel}{\thesubsubpart.}
\renewcommand{\thechoice}{}
\renewcommand{\choicelabel}{(\thechoice).}
\renewcommand{\questionshook}{%
    \setlength{\leftmargin}{0pt}%
    \setlength{\labelwidth}{-\labelsep}%
}
\pagestyle{headandfoot}
\firstpagefooter{}{第 \thepage / \numpages 頁}{}
\runningfooter{}{第 \thepage / \numpages 頁}{}
\begin{document}
    \begin{center}
    \fontsize{15pt}{15pt}\selectfont
    政治學試卷 \textcolor{red}{[Reference solution]} \\
    \vspace{0.1cm}
    班平均: 62.3 分,標準差: 8.4 分,最高: 97 分,最低: 23 分。\\
    \end{center}
    \vspace{0.3cm}
    \fontsize{14pt}{14pt}\selectfont
    壹、是非題 (20 pts)
    \begin{questions}
    \question[2] \textcolor{red}{F} 授課時間為週日。
    \question[2] \textcolor{red}{F} 授課教師有五位。 
    \question[2] \textcolor{red}{F} 授課教材為阿拉伯文。
    \question[2] \textcolor{red}{F} 授課年級為三年級。
    \question[2] \textcolor{red}{F} 授課教室位於地下二樓。
    \question[2] \textcolor{red}{F} 授課教室牆壁顏色是粉紅色。
    \question[2] \textcolor{red}{F} 授課教室沒有白板。
    \question[2] \textcolor{red}{F} 授課教師以黑板授課。
    \question[2] \textcolor{red}{F} 授課教師不是男性。
    \question[2] \textcolor{red}{F} 授課教室沒有電風扇和冷氣。
    \end{questions}
    \vspace{1.5cm}
    貳、解釋題 (30 pts)
    \begin{questions}
    \question[10] 民族自決、民主化。
    \textcolor{red}{
        \\
        民族自決: 一個地區之民族或國族實施自決的權利。 \\
        民主化: 政權由獨裁體制轉變成民主體制的過程。
    }
    \question[10] 保守主義、單邊主義。
    \textcolor{red}{
        \\
        保守主義: 促進和保護傳統的社會制度和實踐的文化、社會和政治哲學。 \\
        單邊主義: 國家、政黨、公司等組織進行單方面行動的任何學說和思想。
    }
    \question[10] 自由主義、專制主義。
    \textcolor{red}{
        \\
        自由主義: 以自由作為主要政治價值的一系列思想流派的集合。 \\
        專制主義: 統治權力集中在單一個體身上。
    }
    \end{questions}
    \vspace{1.5cm}
    參、申論題 (50 pts)
    \begin{questions}
        \question[] 老子道德經第十八章:「大道廢,有仁義;智慧出,有大偽;六親不和,有孝慈;國家昏亂,有忠臣。」
        \begin{parts}
            \part[10] 請您釋義。 \\
            \textcolor{red}{廢棄自然並強調人的真情實感、義理、規範,智慧明照出來,人間的造作詐偽即群起而生。父子、兄弟、夫婦,六親無法和諧共處得強調孝道與慈愛。而國家昏亂才有忠臣展現。}
            \part[15] 若您是提出民族自決的美國總統,如何以民族自決論點批判道德經第十八章。
            \textcolor{red}{
            \\
            聖人之治無法維持以仁義等做為補救。崇尚巧智之時代虛偽狡詐層出不窮。孝慈在六親不和睦時比較得出誰孝順、慈愛。忠臣在國家危亂時比較出誰在忠心為國。
            老子提及維持整體的道德值得追求恆久幸福。 \\
            而仁義智慧孝慈忠在整體失道時被襯托出短暫之道德殘影。 \\
            民族自決: 一個地區之民族或國族實施自決的權利。 \\
            1. 老子提及自決權利? 應為否。(2 pts),理由。(7 pts) \\
             \textcolor{blue}{註: 本題寫是得 1 分。} \\
            2. 五常是否為維繫民族之準則,若否應寫出理由。(5 pts) \\
            3. 獨立問題。 (1 pts)
            }
        \end{parts}
        \question[25] 意識形態經常暗中作祟,以本國大選為例說明之。
        \textcolor{red}{
        \\
        1. 中共介選理由、目的、方式、希望的結果。 (5 pts) \\
        2. 國內兩大主要政黨 (國民黨、民進黨) 之的意識形態異同。 (4, 4 pts) \\
        3. 說明。 (7 pts) \\
        }
    \end{questions}
    \vspace{1.5cm}
\end{document}
